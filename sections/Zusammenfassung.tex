\section{Zusammenfassung}
In der Zeit vom 9. bis 15. Mai 2011, und darauf bestehe ich, da es schließlich keine Information von geringem Interesse ist, war ich weder Schüler noch Schülerin einer Institution die gemeinhin unter der Bezeichnung "allgemeinbildende Schule" verstanden wird.
Ich besuchte gelegentlich eine der Klassen 11 bis 13 - zum Brötchen vorbei bringen. Und zum lernen.
Selbstverständlich habe ich meine Jugend nicht verschwendet und mich stets um gute Noten bemüht, so dass ich einen allgemeinbildenden Schulabschluss in der Tasche habe.
Und dann fragen Sie mich ja noch nach dem "höchsten allgemeinbildenden Schulabschluss", den ich habe. Wenn ich das lese, dann stellen sich mir ganz grundsätzliche Fragen. Warum fragen Sie nur nach dem höchsten? Und überhaupt: Abschluss. Das hört sich doch so endgültig an. Dabei geht's doch danach erst richtig los. Los in die Tretmühle des deutschen Berufsalltags, wenn man denn überhaupt das Glück (um nicht zu sagen: die Gnade) hat, eine Ausbildungsplatz zu bekommen oder genug Geld (oder ausreichend finanzstarke Eltern, Geschwister, Großeltern, Tanten und Onkels) zu haben, die einem ein Studium finanzieren (zuzüglich Studiengebühren, je nachdem in welchem Bundesland man denn zu studieren gedenkt). Also bei mir ist's die allgemeine oder fachgebundene Hochschulreife, zu dem ich es - in der Art und Weise, wie Sie es von mir wissen wollen - gebracht habe. Wie auch immer. Da habe ich 'ne Menge gelernt. Doch viel mehr habe ich erst später bzw. jenseits der Schule gelernt. Dass es nämlich gar nicht so sehr darauf ankommt, WAS ich lerne, sondern vielmehr WIE. Ich bin der Meinung, dass wir ohne wirklich freie und demokratisch orientierte Schulen keine nachhaltige und innovative Fortentwicklung unserer Gesellschaft, in der wir alle zusammen leben wollen, hinkriegen werden. Doch das ist sicherlich ein anderes Thema.
Da ich mich momentan noch in einer Selbstfindungsphase befinde, weiß ich noch nicht recht, in welcher Fachrichtung ich arbeiten möchte. Daher habe ich noch keinen beruflichen Ausbildungs -oder (Fach-)hochschulabschluss.
Mein höchster akademischer Abschluss ist der an einer Verwaltungsfachhochschule.
Derzeit verdiene ich mein Geld als Buchbinder.
Zu Punkt 46, "Stichworte zur Tätigkeit", gebe ich Ihnen gerne auf die konkrete Befragung für Bücher binden folgende Auskunft: Direkt nach dem Aufstehen (frühmorgens) erledige ich meinen Toilettengang (mal mehr, mal weniger erfolgreich). Die Inbetriebnahme der Kaffeemaschine im Anschluß hat sich aus zeitökonomischer Sicht bewährt. Dann entkleide ich mich und schnüffle an meinen Achseln. Wenn nötig, benutze ich nun Dusche und Shampoo, wenn nicht, Deo und Bürste. Auch das Putzen der Zähne (alle!) gehört zu meinen frühmorgendlichen Tätigkeiten. Nun stecke ich die Nase aus der Balkontür, um mich dem Wetter entsprechend anzukleiden. Damit möchte ich vermeiden, dass ich krank werde und somit unserem Gesundheitssystem zur Last falle. Anschließend trinke ich den nun bereits fertig gebrühten Kaffee und verspeise - je nach Gusto und Hungergefühl - einige Brötchen und / oder Brotscheiben mit frühstückskonformen Brotbeilagen. Vermerk Führungsaufgaben: Im direkten Anschluß an das Frühstück führe ich meinen Hund Gassi... Mein restlicher Tagesablauf ist recht stupide. Ich fahre zur Arbeit, maloche dort 9 Stunden und komme anschließend wieder heim. Dann: Hund Gassi führen (Führungstätigkeit!!!), Abendessen, Fernsehen, so einen Stuß wie hier verfassen und dann noch waschen, Zähne putzen (wieder alle!) und dann ausziehen und ab ins Bett. Samstag und Sonntag füge ich auf einem gesonderten Blatt bei.
